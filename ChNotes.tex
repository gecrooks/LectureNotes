% !TEX encoding = UTF-8 Unicode 
% !TEX TS-program = xelatex

\documentclass[Lectures.tex]{subfiles}
\begin{document}

\section{Further Reading}

% TODO: Check origin of this quotation, add citation.
\emph{I got another quarter hundred weight of books on the subject 
last night.  I have not read them all through.} \\
\hspace*{\fill} William~Thomson (Lord Kelvin) Lecture~IX,~p87


% TODO: Expand annotation of Shannon's paper.


\paragraph{Stochastic Thermodynamics:}
For a recent introduction to stochastic thermodynamics see \citem{Peliti2021a} Other general monographs and reviews include \citem{Evans2002a,Harris2007a,Seifert2012a,Spinney2013a, Van-den-Broeck2015a}.
%The first few chapters of \citem{Crooks1999c} provides a basic introduction to non-equilibrium statistical dynamics (as understood circa 1999).

\paragraph{Foundations -- Thermodynamics and statistical mechanics} Efficiency of heat engines and the foundation of thermodynamics: \citem{Carnot1824a}; First law of thermodynamics: \citem{Helmholtz1847a}; Second law of thermodynamics \citem{Thomson1851a} \citem{Clausius1865a}; Entropy: \citem{Clausius1865a}; Statistical definition of entropy: \citem{Boltzmann1872a, Boltzmann1898a, Planck1901a, Gibbs1902a, Shannon1948a, Jaynes1957a,Jaynes1957b}; 
Foundations of statistical mechanics: \citem{Maxwell1871a, Boltzmann1896a,Boltzmann1898a, Gibbs1902a}.



\paragraph{Information Theory}


% see \citem{Spinney2013a}, and then try reading some of the other reviews in the same volume: \citem{Sagawa2013a,Reid2013a,Gaspard2013a}.


%For reviews and expositions of non-equilibrium single-molecule experiments see \citem{Hummer2005a, Bustamante2005a, Ritort2006a, Ritort2008a}.


%\paragraph{Origins:} 
%Fluctuation theorem: \citem{Evans1993a}.
%Evans-Searles fluctuation theorem: \citem{Evans1994a}.
%Gallavotti-Cohen fluctuation theorem: \citem{Gallavotti1995a}.
%Fluctuation theorem, first use of terminology: \citem{Gallavotti1995b}.
%Jarzynski equality: \citem{Jarzynski1997a}.
%Crooks fluctuation theorem: \citem{Crooks1999a}.
%Hatano-Sasa fluctuation theorem: \citem{Hatano2001a}.
%Feedback Jarzynski equality: \citem{Sagawa2010a}.
%
%
%
%
%
%

%% TODO: Split out separate section on entropy
%% TODO: Expand annotation of Shannon
%
%\paragraph{Foundations -- Microscopic reversibility and detailed balance:} Origins: \citem{Tolman1924a,Dirac1924a,Tolman1925a,Lewis1925a,Fowler1925a}. Discussion: \citem{Onsager1931a, Tolman1938a, Crooks1998a, Crooks2011b}.
%
%
%
%
%
%\paragraph{Experiments:} 
%Single molecule Jarzynski: \citem{Liphardt2002a}; 
%single molecule fluctuation theorems: \citem{Collin2005a,Ritort2006a}; 
%dragged optically trapped colloid particle: \citem{Wang2002a, Wang2005a, Wang2005b};
%Hatano and Sasa with optically trapped colloid particle: \citem{Trepagnier2004a}; 
%optically trapped colloid particle, time varying spring constant: \citem{Carberry2004a, Carberry2004b}; 
%colloidal particle in periodic potential: \citem{Speck2007a};
%colloidal particle in viscoelastic media: \citem{Carberry2007a}; 
%colloidal particle in time dependent non-harmonic potential: \citem{Blickle2006a,Speck2007a}; 
%two level system: \citem{Schuler2005a}; 
%turbulent flow: \citem{Ciliberto2004a}; 
%electric circuit: \citem{Garnier2005a, Andrieux2008a}; 
%mechanical oscillator: \citem{Douarche2005a,Douarche2005b};
%bit erasure, colloidal particle: \citem{Berut2012a};
%theory and discussion: \citem{Hummer2001a}. % Ritort papers here
%
%\paragraph{Analytic model systems:} 
%(Model systems for which the work distributions can be computed analytically)
%Harmonic potentials: \citem{Mazonka1999a};
%two level systems: \citem{Ritort2002a,Ritort2004b};
%ideal gas compression: \citem{Lua2005a, Lua2005b,Bena2005a,Presse2006b} and effusion: \citem{Cleuren2006b};
%Gaussian polymer chains: \citem{Speck2005a, Dhar2005a, Imparato2005a, Presse2006a};
%Joule experiments: \citem{Cleuren2006a}, adiabatically stretched rotors: \citem{Bier2005a};
%charged particles in magnetic fields: \citem{Jayannavar2006a};
%adiabatic compression of a dilute gas: \citem{Crooks2007a};
%
%
%
%\paragraph{Simulations:}
%two-dimensional Ising model: \citem{Chatelain2006a};
%fluctuating lattice Boltzmann model: \citem{Chari2012a}.
%
%
%\paragraph{Bochkov-Kuzovlev generalized fluctuation-dissipation theorem:} 
%\begin{itemize}
%\item Origins: %\citem{Bochkov1977a,Bochkov1977b,Bochkov1979a,Bochkov1979b,
%\citem{Bochkov1981a,Bochkov1981b}. 
%\item Summary: \citem{Stratonovich1994a}. 
%\item Relation to Jarzynski equality and fluctuation theorems: \citem{Jarzynski2007b, Horowitz2008a, Pitaevskii2011a}. % TODO: 
%\end{itemize}
%Also Seifert2007ish?
%% Junk: Kuzovlev2011, Bochkov2013
%
%
%\paragraph{Jarzynski equality:}
%\begin{itemize}
%\item Origins: \citem{Jarzynski1997a, Jarzynski1997b, Crooks1998a}.
%\item Connection to fluctuation theorems: \citem{Crooks1999a, Crooks2000a}.
%\item Strong coupling: \citem{Jarzynski2004b}.
%\end{itemize}
%
%% Work microscopic thermodynamic definition. Gibbs, Tolman, Shrondiger, Jarzysnki, me. Discussion. Jarzynski, others... Peliti2008b
%
%
%
%
%\paragraph{Multivariant fluctuation theorems:} \citem{GarciaGarcia2010a,GarciaGarcia2012a,Sivak2013a}
%
%\paragraph{Excess free energy:}
%The connection between the excess free energy of nonequilibrium ensembles and relative entropy seems to have been independently rediscovered multiple times. \citem{Bernstein1972a}~(Eqs. 44 and 54) defined an ``entropy deficiency'' as the relative entropy of a nonequilibrium to canonical equilibrium. \citem{Shaw1984a}~p37 states that available free energy is the relative entropy, but without detailed discussion. \citem{Gaveau1997a}~p348 notes that the relative entropy to equilibrium state is a generalized free energy, but again without much discussion. 
%%See also \citem{Gaveau2002} and \citem{Gaveau2008}. 
%\citem{Qian2001a} provides a detailed discussion of the definition of free energy away from equilibrium and its expression as a relative entropy. 
%\citem{Hatano2001a} discuss the generalization of free energy out of equilibrium and a generalized minimum work principle for steady states.
%%Honerkamp (2002) (p306-309)~\citem{Honerkamp2002} also makes the connection between relative energy and generalized free energy, apparently independently.
%The connection between excess free energy and reversible work is shown in \citem{Vaikuntanathan2009a} using a Jarzynski	equality like argument. The \index{instantaneous stabilization} instantaneous stabilization procedure for (in principle) extracting the reversible (maximum) work is detailed in~\citem{Hasegawa2010a,Takara2010a} and further discussed in \citem{Esposito2011a}.
%Further discussion and consequences of this interrelation can be found in \citem{Sivak2012a} and \citem{Deffner2012a}.
%%[Kudos: DAS for tracking down many of the early citations to relative entropy and free energy.]
%
%%\todo{Mine Deffner for insights}
%% Who do Esposito2011 cite?
%
%
%\paragraph{Dissipation and the relative entropy between conjugate trajectory ensembles:} A good discussion of this relation is found in: \citem{Kawai2007a}. See also: \citem{Gaspard2004b,Gaspard2004a,Jarzynski2006a,Kawai2007a,GomezMarin2008a,Andrieux2008a,Feng2008a,Horowitz2009b,Parrondo2009a,Jarzynski2011a}
%
%
%\paragraph{Feedback control:} 
%Origins: \citem{Sagawa2010a}; % TODO Add Sagawa2008, significance?
%Generalized Jarzynski (Single loop feedback): \citem{Sagawa2010a};
%Experiments: \citem{Toyabe2010a};
%Multi-loop feedback: \citem{Horowitz2010a,Fujitani2010a}. %TODO: Read
%% Feedback reversibility: \citem{Horowitz2011};
%% More: \citem{Sagawa2012,Abreu2012}. % Todo: Read
%% M. Ponmurugan?
%
%\paragraph{Time's Arrow:} Origin of term: \citem{Eddington1928a}; Useful philosophical discussions: \citem{Price1996a,Albert2000a}. Past hypothesis: \citem{Albert2000a}. Length of: \citem{Feng2008a}.
%
%
%
%\paragraph*{Work and heat} The microscopic definition of work was discussed by
%\citem{Gibbs1902a} (Pages 42-35)  and \citem{Schrodinger1946a} (See the 
%paragraphs found between Eqs. 2.13  and 2.14). However, the importance of this viewpoint was not appreciated until advances in simulation and experimentation made it necessary to carefully contemplate performing controlled perturbations of single, microscopic systems. The microscopic  definition of work was also discussed by: \citem{Hunter1993a, Jarzynski1997a, Crooks1998a, Sekimoto1998a}  and 
%further developed in \citem{Jarzynski1997b, Jarzynski1998a, Crooks1999a,Hendrix2001a}.
%Another clear exposition with a discussion of thermodynamic consistency is found in: \citem{Peliti2008b}.
%% \todo{Incorporate insight and citations from Peliti2008b paper.}
%See also \citem{Narayan2004a,Imparato2007a}.
%% The distinction between ``thermodynamic'' and ``mechanical'' work has created some confusion, see, for example,: \citem{???,???,Crooks2009?}.
%% \todo{Also : \citem{Schurr2003,Scholl-Paschinger2006,Jarzynski2000, Peliti2008a}?}
%% \todo{Clearer citation to Sekimoto1998?}
%
%% TODO: Citation to Chris paper where he found many of these references?


\end{document}
